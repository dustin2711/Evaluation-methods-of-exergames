\chapter{Introduction}

This is a test message for git.

\section{Motivation}
Fitness video games, now referred to as exergames, gained a lot of popularity in the last years, mostly due to the development of cheap and small motion sensors. This is not surprising as they enable to combine the healthiness of sports with the motivating aspects of video games.
"There was moderate quality evidence that AVGs can result in benefits for self-esteem, increased energy expenditure, physical activity and reduced body mass index in children and adolescents who used AVGs in the home environment" \cite{santos2020active}
There are hints on increased self-esteem in boys and reduction of girls’ mental confusion \cite{andrade2020effect}.
They can adress health problem of video gamers as they are at a higher risk of being overweight \cite{melchior2014internet} or frequent online game players who report exercising at a lower frequency \cite{kowert2014unpopular}. Also elder people with less connections to video games can train their balance using exergames \cite{lai2013effects}. Playing Pokémon Go was shown to be associated with statistically significant number of daily steps \cite{khamzina2019impact}. This becomes more impressive considering that it is the third most popular smartphone game of all time \cite{link_pokemongo}. However, there are also studies coming to the conclusion that offering exergames, instead of regular video games, to children does not increase their physical activity under naturalistic circumstances \cite{baranowski2012impact}. 
To further improve exergames, reliable ways to evaluate them are advantageous. The available research mostly focused on pure videos games and serious games. Serious games are all video games which pursue another "characterization goal" aside from entertainment, e.g. fighting cancer cells in Re-Mission \cite{link_remission} was shown to help children with cancer to develop a more positive attitude towards their illness with positive health aspects, as shown in randomized controlled trials \cite{kato2008video}.
Exergames represent a subgenre of serious games with a focus on excersing ie body movements. In comparison to general video games, serious games benefit, due to their double mission, from evaluation methods that consider both serious part and game part plus their balance \cite{caserman2020quality}. Exergames again can benefit from evaluation methods that consider the involved physical activity. This paper will give insights to the possibilities and challenges.
"moderate-intensity physical exercise is good at improving emotional response to negative stimulations" [67]
Lack of physical activity is a significant contributing factor to preventable deaths in the American population \cite{bauer2014prevention}.

\section{Approach}
Three studies were proposed as interesting by the supervisor Polona Caserman who is an researcher in the field of serious games. They acted as a good starting point for snowball sampling.
Other studies were often found using ChatGPT 3.5 with queries in the format of "list some studies about ..." which usually gave a good overview. Although the name of the listed studies were wrong many times, the author name and the release year could be used to find the related paper in Google or Google Scholar. 
E.g. the query "list some studies about to evaluation of exergames" led to these studies \cite{fogel2010effects} \cite{lai2013effects}.
To find evidence for several trains of thought or to answer upcoming questions, key words where put into Google Scholar, e.g. "Heart rate variability emotion" to find more about the connection of heart rate variability and emotions.

"heart rate (exergame OR exercise video game OR active video game)"
"Games User Research"
"video game shooter exertion motion controlled"
Furter more, this paper serves as an overview. To keep concise, when summarizing other papers, most often only key information are included. Terms like "may" and "possible" will hint on possible limitations of the studies referred to. The reader is encouraged to dive deeper into topics of their interest. 

\section{Target}
The target of this paper is to be able to examine how well a exergame works. If such a game works, can be supported by scientific studies or by winning game wards \cite{caserman2020quality} but also by playtime. In exergames, you want the player a) to feel pleasure in the moment and b) reach the desired training effects. 
Pleasure in exergames can be measured in many ways including player assessments, psychophysiological measures or general game metrics like play duration and frequence.
Training effects can be measured best based on outcomes including anthropometric (weight, height) and physiological measures, fitness tests, daily metabolic rates and much more.
In order to maximize training effects, regular exercise is advantageous, e.g. as shown by Heaselgrave for muscle growth \cite{heaselgrave2019dose}. Also appropriate session durations are needed.
This paper will give an overview how pleasure and training effects were measured in the scientific research and what other opportunities are there. The research specifically on the evaluation of exergames will be included as far as it is available. Otherwise, the aspects of video games and physical excersises are examined mostly individually and consolidated later.

\section{Terminology of exergame}
There a multiple variations to the term exergame including 
%{\calligra active video game), {\calligra  fitness video game} or even {\calligra  interactive video game} exertion games, Movement-based sports video games
As occuring in the title, this papers will use the term exergame as its widely used in literature, short and catchy \cite{oh2010defining}. Exergame is a blend word compositing of exercise and game while implicitly referring to video games. In the field of scientific research, there is common ground that exergames are video games. Another aspect is that the games need to involve some sort of exercise or physical acitivity. The difference is made clear by Caspersen who defined \cite{caspersen1985physical} physical activity in 1985 as "any bodily movement produced by skeletal muscles that results in energy expenditure" and exercise as "subset of physical activity that is planned, structured, and repetitive and has as a final or an intermediate objective the improvement or maintenance of physical fitness". So, exercise is the more narrow definition that additionally requires some plan, structure and repitition. Some authors prefer this narrow definition for exergames while most researchers use the broader definition including any physical activity \cite{oh2010defining}. This paper will affiliate and also use the broader definition. That means, games like Pokemon Go will be count as exergame since you engage a lot in moving although you don't do it in a planned way.
% "Exergaming is playing exergames or any other video games to promote physical activity"

Exergames motivate the player to engage in physical activity, often by using modeling by a virtual trainer, feedback on the player's performance and reinforcement \cite{lyons13strategies}.
























% Zeige auf, dass Sport viele gesundheitliche Aspekte hat und daher Bewegung gefördert werden sollte
\chapter{Impact of sport on psychology}
Anxiety reduction was shown after 20-minute treadmill run for experienced but not for inexperienced runners. \cite{hatfield87psychophysiology, boutcher86anxiety}
Exercise can lead to increased beta-endorphin levels causing an analgesic and "feel-better" effect \cite{hatfield87psychophysiology}. Also a muscle relaxation response could be found \cite{devries72electromyiographic, hatfield87psychophysiology}.
Physiological arousal of people with higher aerobic power (the ability of the muscles to produce energy using oxygen) is lower and recovers faster to base level which also effects emotions: People that were made angry in experiments, excercised and then sat for some time, showed more intensive emotions, ie aggresive behaviour, at lower fitness levels \cite{zillmann74attribution}.
Contrary to meditation and music appreciation programs, at long term, exercise led to a faster recovery the electrodermal response after applying different kinds of psychosocial stress \cite{keller84physical} while additionally increasing physical fitness. Still, there are also many studies that did not come to the same conclusion which may also arise from habituation due to using the same social stressors \cite{hatfield87psychophysiology}.
A systematic review concluded that extrinsic motivation may be more influential in motivating individuals to start exercise in the short term, while intrinsic motivation may be more predictive in long-term.\cite{teixeira2012exercise} 
"Regular exercise is associated with emotional resilience to acute stress in healthy adults" \cite{childs15regular}.

When people feel motivated by their own choices and have their basic psychological needs met, they are more likely to have positive emotional experiences during exercise. However, for individuals who are externally regulated, perceiving the intensity of the exercise as high can help sustain a better emotional response. \cite{teixeira16needs}
One study proposed that changes in emotions, stress, and effort would be associated with the intensity of running, with high intensity conditions causing more change and being perceived as more stressful and effortful. This hypothesis was supported by the results from the 1.7 km experiment but not the 5.0 km experiment \cite{kerr2000effects}. As exergames rather use low intensities, prolonged sessions could be advantageous. 

One study compared the effects of Wii exergames and traditional exercise on subthreshold depression in older adults, finding that exergames had a direct negative effect on subthreshold depression, with positive emotions mediating this effect. Self-efficacy was not found to be a significant mediator. These findings suggest that exergames may be a valuable tool in reducing subthreshold depression among older adults and could be considered in its treatment. \cite{li17exergames}

\section{Impact of sport on psychophysiological measures}

\subsection{HRV}
As summarized by Makivic, during exercise, HRV typically shifts toward a higher LF/HF ratio, reflecting increased sympathetic activation and after exhausting excerise, towards a more parasympathetic activation.	\cite{makivic2013heart}

Sympathethic activity was shown to increase Electrodermal Activity but not the RMSSD  parameter of HRV.\cite{boettger2010heart}

\subsection{EEG}
Brain waves measured by EEG showed less latency after exercise-induced changes. \cite{reeves85endogenous} Also alpha brain waves showed 30-40 min increased magnitude after exhaustive tradmil walking which could be interpretes as an increased lack of attentiveness to the enivironmental surround \cite{hatfield87psychophysiology}.

Physical activity leads to increased heart rate \cite{hammond1985normal, boettger2010heart, haskell2007recommendation}, vigorous activity leads to rapid breathing and substantial increase in heart rate \cite{haskell2007recommendation}. This could make the emotion-caused fluctuations harder to detect although research on this area is rare.

\section{Smartwatches}
Energy expenditure: Fitbit Surge HR and TomTom watch(r = 0.62-0.69)
average HR: Apple Watch, Fitbit Surge HR, and TomTom watchr (r = 0.47-0.74)
Peak HR:  all smartwatches  (r = 0.59-0.65).  [58 Pope 2019 Validation of four smartwatches in EE and HR]




















\chapter{Evaluation methods for video games}
\label{ch:evaluationmethods}

There are multiple methods available to evaluate games with different characteristics. The methods can be applied during or after the gaming sessions. They can be subjectively, measured by a human and often the players themselves, or objectively using a device. These distictions create the following table.
 
\begin{table}[]
	\begin{tabular}{llll}
					& Live                                              & Afterwards\\
	Subjective      &                                                   & \\
	- Self-reported & Live questionnaires, think-aloud                  & Questionnaires, interviews,\\ 
					& 													& focus groups, heuristic evaluation \\
	- Observed      & Behavioral observation 							& \\
	Objective       & Physiological measures, 							& \\  
					& eye tracker, game metrics         				&\\                                                    
	\end{tabular}
\end{table}                                             								

\section{Demographic background}

\section{Gameplay metrics}
Gameplay metrics analysis in behavior analysis is a valuable tool for tracking and analyzing user behavior in complex computer games \cite{drachen2015behavioral}. It complements existing methods, provides quantitative data, and enhances understanding of the play experience. Despite challenges, it is widely recognized in the game industry and academia as a valuable approach for evaluating and testing games. Drachen identified three categories: Generic gameplay metrics (eg durations, sessions, location, played part of the game, progress), Genre specific gameplay metrics (eg character movement, interactions, interface usage) and Game specific gameplay metrics.
Marston et al \cite{marston2013play} tried integrating the genre exergames in a video game genre map developed by Fencott in 2012 \cite{clay2012game}, with connections to Sports, (People) Simulation, Puzzle and Turn Based genres. Still, exergames can be imagined to be mixed up with any genre as most games can be enabled to be played with the use of motion detection. Eg with current technology like omnidirectional treadmills, many games that support virtual reality technology can be transformed into an exergame \cite{link_katvr}. A positive side effect is possible increased immersion and flow \cite{wehden2021slippery}.
Wols et al \cite{wols2018game} showed 2018 that in-game play behaviour in MindLight, a game teaching relaxation methods, can predict improvements in anxiety symptoms 3 month later.
In 2011, Kivikangas et al \cite{kivikangas2011developing} presented a tool that could record and map ingame events, ingame recordings, psychophysiological measures and live player feedback, which was also enabled by the tool. It can only be used in the Source engine by Valve Corporation.


\section{Think-aloud protocol}
More frequent players talked much more about their insights and game strategies \cite{blumberg2008impasse}.
Tan et al examined the combination of think-aloud with physiological data. Many interesting physiological responses were not reflected by the think-aloud method and sometimes participants reported interesting experiences that could not me seen in the measures eg confusion.
Was used in Tennis \cite{swettenham2020investigating}.

\section{Heuristics}
Heuristics can be used as a help for developers during the early development \cite{desurvire2009game}. Based on the Heuristic Evaluation for Playability (HEP) that shows some correlation to user experience \cite{desurvire2004using} using 43 items, the Principles of Game Playability (PLAY) was developed. It has  yo/no- items in 3 categories  Game Play, Emotional Immersion and Usability \& Game Mechanics.
Strååt et al \cite{straaaat2015applying} applied the Herzberg’s Two-Factor Theory to label items as motivational or hygienic ie if they are of a intrinsic or extrinsic nature or if they are creating positive feelings or disabling negative feelings \cite{alshmemri2017herzberg}. If their impact is neutral to positive or neutral to negative. A group of 23 participants was used for sorting the items. Heuristics about usability were mostly hygiene factors eg "Players feel in control" and motivator were topics like storyline and immersion eg "The game is balanced with multiple ways to win". All in all, this two-factor-theory is a good reminder to think of certain aspects that effect game quality are more prone to have a positive and a negative impact.


\section{Questionnaires}
One important part to understand player's impressions towards video games are questionnaires as they can gather large amounts of data in a standardized way. One major drawback is that they need some time to complete and thus, are often conducted some time after the game was played. The Game Experience Questionnaire (GEQ) tries to solve this problem using a more concise In-game questionnaire \cite{poels2007game}. 




\subsection{The Player Experience of Need Satisfaction (PENS) Model}
The creators of this questionnaire claim that fun and satisfaction are outcomes of psychological processes and not the processes themselves \cite{rigby2007player}. So in order to create an entertaining game for the player, you need to unterstand the underlying processes und be able to describe the "underlying energy that fuels actions" \cite{rigby2007player}. Thereby, PENS is grounded on the well-established self-determination theory from the 1980s. Ut elicited validated measures in many fields \cite{pietrabissa2020development,lohmann2017measuring,richards2021further}. This theory suggests that there are three basic psychological needs \cite{deci1985intrinsic}:
\begin{itemize}
	\item Autonomy: The need to feel in control of one's actions and have the freedom to make choices.
	\item Competence: The need to feel capable, effective, and skilled in one's pursuits or endeavors.
	\item Relatedness: The need to feel connected, supported, and engaged in meaningful relationships with others.
	\item -
	\item Presence/Immersion: emotional engagement in the game
	\item Intuitive Controls
\end{itemize}

Interestlingly, two of the three main factors, competence and relatedness, seem to be associated with positive emotions in long-term sport \cite{wienke2016qualitative} (like novelty seeking and physical exertion).

A study from 2018 found large support for the PENS \cite{johnson2018validation} and its categories presence, autonomy and relatedness. It proposed conclusion of Intuitive Controls into the Competence factor, especially if the players are more experienced with the game.

"the literature provides good evidence for the value of SDT in understanding exercise behavior". " intrinsic motives (e.g., challenge, affiliation, enjoyment) were positively associated with exercise behavior". "one study/sample performing correlational analysis to explore the links between health motives and exercise". "As expected from theory, controlled motives (social recognition, appearance/weight) did not predict, or negatively predicted, exercise participation". \cite{teixeira2012exercise}.
The disadvantage of high organizational effort that often goes with exercises \cite{mullan97variations} are not such a big deal in exergames.


\subsection{The Player Experience Inventory (PXI)}
Abeele et al created the Player Experience Inventory (PXI) to measure, analyse and understan a player's experiences \cite{abeele2020development}. It is split in immediate experiences based on game design choices and emotional experiences like immersion and mastery. Items were examined with the use of 64 Games User Research (GUR) experts, validated and evaluated with studies on totally 529 participants. They came up with 10 factors with 3 items each: Meaning, mastery, immersion, autonomy, curiosity, ease of control, challenge, progress feedback, audiovisual appeal and goals \& rules. The inventory was set in relation with the PENS and the AttrakDiff, a questionnaire to measure hedonistic and pragmatic quality in a broder area and not just for games \cite{hassenzahl2003attrakdiff}. Here are the Pearson correlation coefficients:
\includegraphics[width=\linewidth]{"PXI_vs_PENS.png"}
A good discriminant validity showed that the factors were shown to be indeed sufficiently unrelated.


\subsection{Self-Assessment Manikin (SAM) scale}
The SAM is a tool to measure the emotional affective reaction in the dimensions pleasure, arousal, and dominance by letting the user point on pictures. [70] The Self-Assessment Manikin was used during balance exercises to measure the emotional state [69]. SAM was shown as a reliable and objective way to measure emotions in VR Gaming [71].

\subsection{Game Experience Questionnaire (GEQ)}
The Game Experience Questionnaire was developed 2007 by a European research project named FUGA \cite{poels2007game}. It is widely used in at least 515 papers \cite{law2018systematic}. The structure was examined by using focus groups of infrequent and frequent gamers talking about their feelings on video games and experts building on these results and theoretical considerations. A questionnaire was created assessed by a 5-point Likert-scale from "not at all" (0) to "extremely" (4). Orignally ten factors and 92 items were reduced after the questionnaire was tested with 380 participants referring to a game of their choice. After exploratory factor analysis on the results, a 7-factor solution with 82 items was concluded. It explained 52 \% of variance in all items with most questions havinge more than 30 \% correlation to only one factor. The seven factors are:
\begin{itemize}[noitemsep]
	% \itemsep0em
	\item Immersion (previously two factors Sensory and Imaginary Immersion)
	\item Tension (said to unforseenly emerged from Negative Affect although the factor Suspense already existed)
	\item Competence (includes previous factor Experienced Control)
	\item Flow
	\item Negative Affect
	\item Positive affect
	\item Challenge
\end{itemize}
Two factors "Connectedness" and "Negative affective experiences related to playing with others" were shifted to an own questionnaire, the "social presence module" including another new category "behavioural involvement" .

A paper from 2018 did a systematic review using 73 studies that applied the GEQ \cite{law2018systematic}. Many critic points were found for both a GEQ itself but also for the studies that used it.
\begin{itemize}
	\item Only 31 papers of 73 offered an explanation why the GEQ was used (31 because it is validated, 10 due to popularity, 8 due to multidimensional structure and 6 for being theoretically and empirically founded) 
	\item 47 Papers did not state the number of questions used whiel the GEQ offers a 33- and a 42-items version 
	\item Often times, papers used a selection of factors or items or adapted the items with mostly missing reasoning, partially explained by overlapping after using factor analysis 
	\item The used scale was not reported by 40 papers, 27 did not use the origin scale and 3 of them even used a mix of 5 and 6 answer options instead of 5 
	\item One main critic point was that there are two versions of the GEQ, from 2007 and 2013 with none of them formally publicated. The 73 analyzed papers referenced the 2007 version only once and the 2013 version not a single time. Often times, the GEQ was cited as "Manuscript in preparation" 
	\item 17 papers stated Cronbach's $\alpha$ for internal consistency of the factors with following values:
	\begin{itemize}
		\item Flow and Competence: 0.7 to 0.94
		\item Positive Affect and Immersion: 0.49 to 0.85
		\item Negative Affect, Challenge and Tension: 0.3 to 0.74
	\end{itemize}
12 of these 17 papers stated low internal consistency, especially for challenge and negative affect 
No paper could reproduce the 7-factor structure but six proposed using six factors with Negative and Positive Affect being summarized as "joy" 
\end{itemize}

Finally, a validation study was conducted with 633 participants in an online survey (age = 33 47 ± 10 57, game experience = 19 5 ± 8 9 years) using the 33-items GEQ. A confirmatory factor analysis condluded that the factor Negative Affect is not satysfying due to a low reliability coefficient and the factor Challenge just barely. A exploratory factor analysis was conducted and came to the conclusion to remove some items that did not load well on the given or any other factor. Renaming Flow to "loss of time" was proposed. The Challenge factor only has 3 remaining items with need for filling up. Also Tension and Negative Emotion should be combined to one factor which results in these factors:
\begin{enumerate}
	\item Immersion
	\item Competence
	\item Flow
	\item Negative Affect
	\item Positive affect
	\item Challenge
\end{enumerate}

Unfortunately, the reviewing paper does not adress multiple issues.
What are possible readons why the results of the conducted study differ from the original study?
Are there maybe missing questions that could lead to new factor that the GEQ is missing?
Can the items of Negative and Positive Affect be summarized like in six reviewed papers? (with inverting the questions for Negative Affect)
Can the factor Challenge be added to the factor Competence with inverted questions?

Another validation study use exploratory and confirmatory factor analyses to find partial support for the GEQ and large support for the PENS \cite{johnson2018validation}, also proposing to summarize negative affect, tension/annoyance and challenge to one single factor negativity.
	
Another trivial problem of the GEQ is that there is another questionnaire in the field of game research with the same abbreviation GEQ. The Game Engagement Questionnaire also measures game experience, but rather the psychological engagement \cite{brockmyer2009development} using 19 items in four categories absorption, folow, presence and immersion. One item from the presence category is e.g. "Things seem to happen automatically". The development was supported by a test to distract participants playing a video game with a prerecorded statement and classifying their reaction. They found a hint on a negative relationship between the Game Engagement Questionnaire and the Dissociative Experiences Scale (DES) that measures dissociation eg that you find yourself in places without knowing how to got there \cite{wright1999measuring}.

\subsection{Flow State Scale}
The term flow was developed to understand the phenomenon of intrinsically motivated activity \cite{nakamura02concept}. It was described as the ideal experience during work or play \cite{csikszentmihalyi1992optimal} In flow, a person is totally invovled in a task while challenge and skill level hold a perfect balance \cite{jackson1998psychological}. Jackson et al have further shown that, when experiencing flow, the challenge level was perceived and rated slightly higher with a value of 8.3 compared to the skill level with a value of 7.3 \cite{jackson1996development}. Getting in flow state would be pleasant for exergames too also because it plays a role in motivation and enjoyment and people seem to show peak performance while in flow state \cite{jackson1996development, harris20review}.
To evaluate flow, there is a questionnaire called the "Flow State Scale". It originally consisted of 54 items with 9 factors but be reduced to 36 items while keeping a high reliabiliy with coefficient alphas rated above 0.80 \cite{jackson1996development} 
Here are the 9 original factors. A study with 1083 athletes proved an acceptable fit \cite{stavrou11confirmatory}. THe last two have the most potential for removal \cite{jackson1996development}:
\begin{enumerate}
	\item Sense of Control
	\item Challenge-Skill-Balance
	\item Clear Goals
	\item Merging with task
	\item Unambigious feedback
	\item Concentration
	\item Loss of Self-Consciousness
	\item Transformation of Time (weakest factor, also shown to not have acceptable internal consistency \cite{vlachopoulus00confirmatory})
	\item Intrinsic motivation (may be more of a higher-order factor)
\end{enumerate}

Another study developed a flow state scale for occupational tasks in occupational therapy by letting the participants play video games and confirming that the computer game task used in the study represented measurable activities typically encountered in occupational therapy. By using a confirmatory factor analysis, they condluded three factors Sense of control, Positive emotional experience and Absorption by concentrating.
The total score on their flow state scale was negatively correlated with the total score on the State-Trait Anxiety Inventory (STAI), indicating that the flow state was associated with lower anxiety levels.







\chapter{Evaluation methods of exercises}

\section{The Borg Rating of Perceived Exertion scale}
The The Borg Rating of Perceived Exertion (RPE) scale was developed by Gunnar Borg in 1982 \cite{borg1982psychophysical} "is a tool for measuring an individual’s effort and exertion, breathlessness and fatigue during physical work and so is highly relevant for occupational health and safety practice" \cite{williams2017borg}. It is very often used in research with 950 highly influential citations regarding to semanticscholar \cite{link_borginfluence}. In its original form its a scale from 6 (no exertion at all) to 20 (maximum exertion) with 1 point each representing 10 hearts beat per minute in a halthy adult \cite{williams2017borg}, so 6 refers to a resting heart rate of 60 \cite{link_borgrating}.
It is used "particularly in the field of sports medicine, where it is used by trainers to plan the intensity of training regimes, and in the workplace, where it is used to assess the exertion used in manual handling and physically active work" \cite{williams2017borg}.
"over the course of a working day, high neck muscle tension correlated well with high perceived levels of physical exertion" \cite{williams2017borg, jakobsen2014borg}

"The Borg RPE scale has been compared with other linear scales such as the VAS and Likert scales. The sensitivity and reproducibility of the results are broadly similar although work by Grant et al. [13] suggests that the Borg may outperform the Likert scale in some scenarios." \cite{williams2017borg}


\section{Metabolic Rate}
Wang et el (2016) \cite{wang2006metabolic} showed significant increases in various metabolic and physiologic variables during video game play for 20 boys aged 7 to 10.



















\section{Biofeedback}


\subsection{Accessing emotions}


\subsection{Applicability in videos games}
Psychophysiological measures can be a helpful way in evaluating exergames because they "provide an objective, continuous, real-time, noninvasive, precise and sensitive way to assess the game experience" \cite{kivikangas2011review}. 
" Some of the earliest are from Mandryk and Inkpen (2004) and Hazlett (2006), who have presented studies (albeit with small sample sizes) supporting the use of psychophysiological measures in game research. More recently, Nacke (2009) and others have published studies as an attempt at a common methodology for a design-oriented approach. Their papers evaluate EEG (Nacke et al. 2010b), EDA, HR (Nacke 2009; Drachen et al. 2009) and facial EMG (Nacke 20" \cite{kivikangas2011review} "clearly showed the necessity of proper experimental design and that care must be taken in interpreting the signals: otherwise, for instance, self-reported and physiologically indexed emotions may turn out to assess significantly different things"  \cite{kivikangas2011review}
"All these studies demonstrate that physiological signals are closely related to players’ self-reported emotional states and behaviour, while Chanel and others (2011) found support that the fusion of several physiological modalities increases the recognition accuracy. This shows that multiple measurements are still needed for a reliable interpretation of the player experience"  \cite{kivikangas2011review}

\subsection{Applicability in exercises}
Another important point for exergames is portability and applicability during excercise. There may be games with heavy movement where a small or wireless measurement device may be advantageous 
Now comes an overview of measure possibilities. For more details, the just mentioned paper by Nacke 2014 is recommended 




"A large number of studies have shown that psychophysiological measures can be used to index emotional, motivational and cognitive responses to media messages" \cite{kivikangas2011review}
"physiological signals could be used, for example, at the player test phase to identify strong emotional episodes and compare them to expectations, or to control the successful emotional elicitation of game event designed to be emotionally arousing" \cite{kivikangas2011review}
The human body offers many ways to gain insights into someone's emotions state, e.g. heart rate and skin conductance together can be used to effectively recognize some basic emotions \cite{hamdi2015emotion}. Also, these measurements can be taken continously during the gameplay and can be mapped to a recorded gameplay session afterwards \cite{nacke2015physiological} which allows great post-analysis. Nonetheless, they should not be relied on alone. The researcher should always ensure the correct interpretation of the signals \cite{nacke2015physiological}, e.g. by validating them using other measurement methods like questionnaires 
One major drawback of these measures is that they are not only be affected by the game experience itself. Any movement with the body affects physiological measures, e.g. physical exercise results in heart rate increase \cite{javorka2002heart}. This can be obviously a problem in evaluating exergames where the body response usually consists of a psychological and a physical reaction. Aside from moving, also any stimulus in the environment that is not gameplay-related can be disturbing and affect the meausrement \cite{nacke2015physiological}. Also the demographic background of any person and the baseline of the measured variable in idle state should be taken into consideration \cite{nacke2015physiological} 

\subsection{Circumplex model of affect}
Although it is often hard to directly map physiological measurements to basic emotions, Russell’s circumplex model of affect \cite{russell1980circumplex} can be applied to simplify things. It uses just two dimensions. The first dimension valence measures how pleasant or unpleasant something is experied. The second dimension arousal emasures the level of stimulation. Remington et al (2000) \cite{remington2000reexamining} showed that the model has an acceptable fit and its circular structure is supported. However, there were also found negative correlations for emotions of opposite sides of the circle, with a dependancy of the assessment situation. Evmenenko et al (2020) published a systemativ review about the model in contexts of physical activity and identified no major issues with its usage. They found that the studies mostly operated in all quadrants except in the unpleasant/low-stimulation quadrant.
 Positive and negative emotional valence can be assessed by using facial EMG \cite{kivikangas2011review}. They can be more easily measured and concrete emotions can be derived from them \cite{seo2019automatic} 






\subsection{Facial Electromyography}
Facial Electromyography (EMG) uses sensors are attached to the skin to measure the eletric and so physical activity of muscles. They can be used to detect facial expressions. It was shown that eye \cite{ravaja2018phasic} and eyebrow muscles are a good indicator for positive and cheek muscles a good indicator for negative emotions \cite{nacke2015physiological,mandryk2006using}.
Hazlett conducted a study which showed, that during a car racing game, the zygomaticus muscle, which controls smiling, correlated well to positive events and corrugator muscle, which controls frowning, correlated well to negative events.
\cite{hazlett2006measuring}
Disadvantages are discomfort wearing face sensors, the inability to talk during gameplay and unnatural facial expressions due to feeling the sensor \cite{nacke2015physiological}
Also computer vision techniques can be used to obtain facial expressions but when discovering smiling, "The results showed that EMG has the advantage of
being able to identify covert behavior not available through
vision. Moreover, CV appears to be able to identify visible
dynamic features that human judges cannot account for" \cite{hernandez19invisible}

\subsection{Electrodermal Activity}
Electrodermal Activity (EDA) uses two electrodes to measure the conductivity of the skin, e.g. between two fingers. This is correlated to the sweat gland activity and that again is linked to emotional arousal \cite{nacke2015physiological, dawson2017electrodermal}. The measuring device can function wireless and be made out of soft materials \cite{kim2021soft} do not be disturbing during exergames 
The main disadvante is some seconds of latency which makes it more difficult to link a signal to a cause \cite{nacke2015physiological} 

"Significant covariation was obtained between (a) facial expression and affective valence judgments and (b) skin conductance magnitude and arousal ratings" \cite{lang93pictures}
"most drivers studied, skin conductivity and heart rate metrics are most closely correlated with driver stress level" [66]


\subsection{Heart Rate}
Heart rate (HR) as primary parameter of cardiac activity was shown to effectively recognise all four basic negative emotions anger, fear, disgust and sadness \cite{levenson2003blood} and is tied to emotional arousal \cite{nacke2015physiological}. There are many possibilities to measure heart rates. Chest strap monitors are very precise and are affordable (r = 0.996 in comparison to EEG, 75 €) \cite{link_herzfrequenzsensor} and also current smartwatches can deliver good results (r = 0.92, 350 €) \cite{gillinov2017variable}.
"Over the past four decades, heart rate has been used to assess emotional arousal in non-human animals in a variety of contexts, including social behaviour, animal cognition, animal welfare and animal personality" \cite{wascher2021heart}

\subsection{Heart Rate Variability}
Heart Rate Variability (HRV) can be also obtained from measuring the heart rate but needs further analysis \cite{nacke2015physiological}. There is emerging analysis for its role in regulated emotional responding \cite{appelhans2006heat} with higher variability typically correlating with better emotional regulation [source?]. However, it was shown that heart reate variability alone is not that effective in recognizing emotions with a baseline below 50 \% \cite{ferdinando2014emotion}.


\subsection{Finger Pulse Volume}
Finger Pulse Volume (FPV) was shown to be sensitive to changed in experimentally created anxiety \cite{bloom77finger}.

\subsection{Electroencephalography}
Electroencephalography (EEG) is used to measure brain waves with a high temporal accuracy \cite{nacke2015physiological}. Some consider it as "ideal psychophysiological measure" as its "constantly fluctuating and continuously responsive to changing psychological stimuly" \cite{hatfield87psychophysiology}. The activity of different brain regions can be visualized but there is a lot of interpreation possibilities, especially since the origin of the brainwaves is not apparent \cite{nacke2015physiological}. At least it can be used to measure a player's engagement in a video game, e.g. there was increased activity when diing in. Super Meat Boy \cite{mcmahan2018evaluating}. With the Emotive EEG headset \cite{link_emotiv}, there is also portable technology available.
"Relatively greater left frontal activity indicates a propensity to approach a stimulus, whereas relatively greater right frontal activity indicates a propensity to withdraw from a stimulus (Davidson 1998). It must be emphasized that
frontal asymmetry is not a measure of positive or negative affect per se, but it taps a broader motivational tendency towards approach-related or withdrawal-related behaviours and emotions (Allen et al. 2001; Davidson 1998)" \cite{kivikangas2011review}
Alpha waves were found to be more prominent during tasks that did not require attention, such as mental arithmetic while beta waves were associated with emotionally positive or negative tasks, as well as cognitive tasks \cite{ray85eeg}.


\subsection{Respiration rate}
Studies reported higher respiration rate during arousal \cite{homma2008breathing}, increases in inspiratory time when laughing and decreases during disgust \cite{boiten1998effects}. Although this measure does not give as much insights as others, it can be well estimated from using heart rate sensors, so no addtional hardware is neccessary when heart rate is already measured. The root mean square error then is 0.648 per minute \cite{natarajan2021measurement}.

\subsection{Eyes}
Lu et al (2015) \cite{lu2015combining} examined the accuracy oft detecting positive, negative and neutral emotion using eye movement, EEG and a combination of both. The accuracy using eye movements was 77.80\% and using EEG 78.51\%. Combining both methods with fuzzy integral fusion, as the most effective method, resulted in an accuracy of 87.59\%. Just for the eye movement, the researchers used 33 different features including pupil diameter, dispersion, fixation, blinking and saccade. Unfortunately, the paper does neither provide accuracies for the eye measure system nor which setup was used. It remains unclear if concentional cameras or specialized equipment was used.
However, already in 2006, Böhme et al \cite{bohme2006remote} built an eye tracking setup with an average accuracy of 1.2 degrees. They used a single 1280x1024 pixels camera and two infrared LEDs to generate reflexes on the eyes' surface. An image processing algorithms processed the images. The tolerance for head movements were 20 cm along all three spatial axes. Regarding the progress in camera and image processing technology, it is currently highly possible, that an affordable eye tracking system can be set up to apply the methods developed by Lu et al.
Valliappan et al \cite{valliappan2020accelerating} created a method to track eyes with the smartphone front camera with an accuracy of 0.6 to 1.0 degrees.

\subsection{Body}
Ekman et al expressed that upper body movement and position, measured by acceleration sensors or cameras, can play a role in attention, interest and emotions\cite{ekman2012social}. Walbott \cite{wallbott1998bodily} found that some evidence suggests specific body movements accompany certain emotions, differences in movement can be attributed to activation levels rather than emotion specificity, indicating a need for further examination.

\subsection{Physiological linkage} (TO DO)
Ekman et al defined the concept of physiological linkage "by gathering simultaneous psychophysiological measurements from several players" \cite{ekman2012social}. It can be used to assess social presence and social interaction in different types of media, including games. It allows for comparing playing contexts and determining the level of social presence in various forms of social play. Sequences of physiological reactions can provide insights into the bidirectionality of interaction and identify the originator and recipients of emotions, offering a deeper understanding of social dynamics. However, further research is needed to validate and refine the use of physiological linkage, considering factors such as mediated environments and predictive validity. The choice of methods should be tailored to the specific case, and linkage measures should be complemented with other methods, such as self-report. Game structure and communication patterns should be taken into account when analyzing physiological data, and social presence should be considered in conjunction with individual experiences. .


















\chapter{Measuring exergame outcomes}

\subsection{Study design}
Field > expertimental
randomized
Controlled (conventional intervention and/or no intervention)
Häufige Proben (short and weeks after intervention)


"The [metastudy] results show that playing AVGs significantly increased heart rate, VO2 (oxygen consumption), and EE (energy expenditure) from resting" \cite{peng2011playing}

Russell 2008 examined short-term psychological effects of interactive video game exercise in young adults, comparing them to traditional exercise. He assessed mood 10 minutes after acitivity. Traditional exercise resulted in highest positive mood compared, followed by exergame activity and eventually sedentary activity. \cite{russell2008short}.

\subsection{Metastudy results}

Biddiss et all (2010) \cite{biddiss2010active} found that exergames can enable light to mdoeratore physical activity but did not find much evidence for long term effects. They investigated 18 studys that included "mainstream" exergames. Participants needed to be younger than 21 years and the study needed to include measures of energy expenditure, activity patterns, physiological risks and benefits, enjoyment and motivation; 
 Activity levels: 222 += 100 \% energy expenditure
 Heart rate: 64 += 20 \%





\subsection{Measures}
The effectiveness of exergames can be evaluated by comparing their outcomes with conventional interventions. Many evaluations methods and dependant variables are conceivable 

\subsection{Energy expenditure}
Game Play in 7- to 10-Year-Old Boys showed a MET of 1.4 \cite{ainsworth20112011}.
In 2012, Lyons et al \cite{lyons2012novel} researched the energy expenditure of 100 young adults during playing different game modes in Wii fit. The energy expenditure was calculated from expired oxygen estimates. In the following stated in kcal/kg-1/h-1, it was 4.45 +- 1.58 for aerobic games (like jogging) and 1.75 +- 0.46 for balance games (like skiing), with slightly smaller values for overweight people. The Compendium of Physical Activities \cite{ainsworth20112011} collects data from multiple studies and also included values from exergames as you can see in Table ???. A MET of 1 is defined historically by a somehow average energy expenditure while sitting still.
In a study with 12 adults (age 18 to 53, BMIs 18 to 37), Noah et al (2011) \cite{noah2011vigorous} found out that playing Dance Dance Revolution is capable of providing vigorous exercises. During a 30 minutes session at "Heavy" difficulty with short breaks after each song, a mean energy expenditure of 8 METs, a heart rate of 157 bpm and 9 kcal/min were measured. These values are high enough to maintain physical fitness while the participants stated that the game is enjoyable. 


\includegraphics[width=\linewidth]{"MET-physical-activities.png"}

A meta analyses with 17 studies showed similar anxiety level reduction compared to conventional interventions, but mostly no improvement as additional treatment to conventional interventions \cite{viana2020effects}. This evidence can be limited as 93\% of studies did not report when the anxiety levels were assessed after the last exercise session. The anxiety was measured most often using the Hospital Anxiety and Depression Scale (HADS-A) followed by the State Anxiety Inventory (STAI) and Anxiety Inventory (BAI).

The same authors did a between-­groups meta-­analyses to examine the effect on muscle strength of exergames versus no intervention and usual care intervention \cite{viana2021effects}. This was defined as "conventional rehabilitations, home-­based physical activity programs, resistance training, balance training, traditional multicomponent training, and other similar programs".
Comparing exergame vs. no intervention showed no effects in handgrip strength in heathy/unhealthy middle-­aged/older adults, knee extension maximum voluntary isometric contraction (MVIC) in healthy older adults. Exergames vs. usual care showed advantages in improvement of handgrip strength, knee flexion MVIC, elbow extension MVIC and propably upper and lower limb muscle strength, but no advantages for knee extension MVIC or elbow flexion MVIC. When exergaming was used additionally to usual care, it only provided improvement of handgrip strength in children with hemiplegic cerebral palsy.

This review systematically examined the role of exergames in reducing weight-related outcomes among obese children and adolescents \cite{valeriani2021exergames}. 10 studies met the inclusion criteria. e.g. no studies with participiants with other ongoing weight-losing interventions ang high quality regarding the tool for Quality Assessment of Controlled Intervention Studies designed by the National Heart, Lung, and Blood Institute’s (NHLBI) \cite{link_studyquality}. Seven out of ten studies reporting better outcomes in the exergame than in the intervention groups. While there is a potential positive effect further research is needed to determine their effectiveness in childhood obesity treatment and identify the most effective approach.  Weight-related outcomes were: weight/body mass index/z-scores, fat and/muscle mass and hip and waist circumference. 
Another review analysed the impact for childhood obesity in field-based studies \cite{gao2014field} as naturalistic environments are advantageous in order to prove practical benefit \cite{baranowski2012impact}. It came to unclear results for both physical activity and weight-related outcomes due to issues regarding design, measurement and methodology. The results show that the examined studies often had small sample sizes and short durations or low frequencies per week, limiting the generalizability and practical implications of the findings. It is noteworthy that the inclusion of children and adolescents in all studies may be influenced by the availability of exergames designed for this age group and the rising concern over childhood obesity in the USA.

In another three month study with 117 college student \cite{huang2017can}, exergames lead to improvements in normalized diastolic blood pressure, sit-up tests and 3-min step tests but body fat percentage was increased. Regular exerciser showed push-ups improvement and irregular exercises improved response time.

"Being provided with the opportunity to engage in DDR or other exergames, children and adolescents became more genuinely interested and selfefficacious compared with conventional physical education classes such as fitness or aerobic dance "

"For example, it was found that EE values of DDR were equivalent to 7.0 metabolic equivalents (METs), which is in the moderate-to-vigorous intensity PA (MVPA?)" \cite{gao2014field}.


dual-energy X-ray absorptiometry [\%fat and bone mineral density {BMD}] and magnetic resonance imaging. Cardiovascular risk factors included blood pressure, cholesterol, triglycerides, glucose and insulin.
Exergaming reduced body fat and increased BMD among those adolescent girls \cite{staiano2017randomized}.

In a 3-month-intervention study, several effects of exergaming on 29 older adults were examined \cite{neumann2018effects}. In the outcomes measures, enduraced assessed by a 2 minutes step test increased significantly, moderate effects were measured in quality of life and lower body strength, small effects were detected in gait speed, mobility in the lower body and the balance capabilities.

SENIOREN AUF TANZMATTE \cite{peng2020novel}
"The Senior Fitness Test was used for assessing various dimensions of functional fitness, including a 30 s chair stand test for lower-limb muscle strength, 30 s arm curl test for upper-limb muscle strength, 2 min step test for aerobic endurance, chair sit and reach test for lower-body flexibility, back scratch test for upper-body flexibility, and 2.44 m up and go test for agility and dynamic balance. The test–retest reliability of the Senior Fitness Test was reported to be high to very high in a normal older population"
"The foot tapping test (FTT), with good reliability (ICC = 0.79) (Hinman, 2019), is a simple test used to examine lower-limb motor function"
% "The Montreal Cognitive Assessment (MoCA), with good reliability (Cronbach’s α = 0.86 and ICC = 0.88) and validity (r = 0.91 between MoCA scores and Mini-Mental State Examination scores) (Tsai et al., 2012), was used to assess cognitive function including attention and concentration, executive function, memory, language, visuoconstructional skills, conceptual thinking, calculations, and orientation."
% Dual-Task Walking: "In line with a previous study (Liao et al., 2019), the DTW testing protocol comprised two conditions: (1) walking while counting backward in increments of three from a random number between 90 and 100 (cognitive DTW, DTW_C) and (2) walking while carrying a tray (size: 38 cm × 28 cm × 5 cm) that was 80% full of water (motor DTW, DTW_M)."
"The self-reported fall risk questionnaire (FRQ), which was developed based on evidence and clinical acceptability and incorporates the strongest evidence-based fall risk factors, has good concurrent validity"

Emotions:
Anxiety

General:
Weight/height/BMI

Muscle strength: MVIC (Maximum voluntary isometric contraction = maximal fest Muskel gegen Objekt anspannen))
isokinetic strength (gleiche Geschwindigkeit, verschiedene Gewichte)
handgrip strength

1RM (1 repitition maximum weight) (4 from \cite{viana2021effects})

Behavior:
Physical activiy (accelerometers)


MOTOR LEARNING
Designing educational exergames should be based on the knowledge of motor control and motor learning mechanisms, considering motor learning as an internal process that evaluates individual ability and performance. Incorporating scientific paradigms of motor control and motor learning into exergame design is crucial for assessing the actual effectiveness of exergames in training and rehabilitation programs. \cite{di2012exergames}

\cite{peng2020novel}
(including upper- and lower-extremity strength and flexibility, grasp strength, aerobic endurance, static balance, dynamic balance and agility), a foot tapping test (FTT), the Montreal Cognitive Assessment (MoCA), DTW, and a fall risk questionnaire (FRQ) 

GPT
***

Physical Health:
****************
x Body Mass Index (BMI)
x Body composition (e.g., percentage of body fat, lean mass)
Cardiovascular fitness (e.g., VO2 max, heart rate)
x Muscular strength (e.g., handgrip strength, leg strength)
Muscular endurance
Flexibility
x Balance and coordination
x Reaction time
Agility
Motor skills

Physical Activity and Energy Expenditure
****************************************
Total minutes engaged in physical activity
Step counts
Energy expenditure (calories burned)
Metabolic Equivalent of Task (Met) (energy expandature relative to mass)
Sedentary behavior (e.g., sitting time, screen time)

Cognitive Health and Function
*****************************
Cognitive performance (e.g., attention, memory, executive function)
Processing speed
x Reaction time
Task performance accuracy
Problem-solving ability
Visual-motor coordination
Dual-task performance
Balance:					Berg Balance Scale
                            Senior Fitness Test.

Psychological and Behavioral Measures
*************************************
Enjoyment and engagement
x Mood and affect
Self-esteem
Social interaction
Adherence and compliance/Übereinstimmung to exercise programs


\subsubsection{Physical activity in inactive children}
A study from 2010 has shown the big potential of exergames for a physical education classroom \cite{fogel2010effects}. By the teacher's observation, four 5th grade class children did not move as much as other children. These "inactive" labeled children should be motivated to move more and participated in the study. 
Contrary to the default lesson where children played e.g. ball games, the exergame lesson consisted of 11 exergame stations. There, the children rotated every 10 minutes and played games like "Fit Interactive 3 Kick", a martial art simulator that uses three foam pads to be hit 
The measured variable was "physical activity" which was defined as moving a large muscle group although it is not stated which device was used for measuring. It was shown that the inactive children increased their physical activity from 1.6 min to 9.2 min during a lesson. This massive increase could be enabled by enhancing the time of opportunities for participating in excercises from 3.8 to 11.6 min. This was mainly reasoned by the fact that the teacher spent way less time providing instructions. 
It is unclear if the effects would persist longer than the given six weeks or it is realted to the novelties. Additionally, the study relies on one single physiological measure as it spared measuring the heart rate in favor of saving time during the lessons. Still, the pure method of comparing exergames with a default setting enabled interesting insights and should be further examined.

\subsubsection{Balance training for elder people}
A study from 2012 evaluated the effectiveness of exergame exercises on balance improvement in older adults \cite{lai2013effects}. After six weeks of intervention, significant improvements were observed in typical balance tests such as the Berg Balance Scale. The positive effects partially persisted even after 6 weeks without further exercises. The exergame approach showed advantages in terms of feasibility and attractiveness compared to conventional exercises, due to instant and positive feedback and the more challenging nature. The study suggests the need for further research to examine the long-term effects and compare the exergame approach with traditional physical therapy.

Riding a bike while watching a bike tour video that gets played faster the faster you cycle was compared with riding a bike to play a video game with way more motivation for the video game. \cite{hardy2011adoption}

\subsubsection{PlayStation EyeToy intervention}
Playing active video games can help overweight or obese children who were current users of sedentary video games to improve their body mass index. This improvement is most likely because their aerobic fitness gets better \cite{maddison12avg}. There were 162 participants in the intervention and 160 in the control group, all having a Playstation 2 or 3 gaming console but no fitness video games. They received the neccessary hardware and a selection of Sony PlayStation fitness video games that use EyeToy, an addtional camera for the console, to detect body movements. They were encouraged to engage in daily 60 minutes of exergaming in periods of inactivity or traditional gaming. The study went 24 weeks and after 12 weeks, a bunch of new games was sent to the intervention group to ensure sustainability.
Outcome measurements were:
BMI (by measuring height and weight)
Body fat was assessed using standardized bio-electrical impedance analysis procedures
Aerobic fitness was assessed using the 20 meter shuttle test [12,13].
Seven day physical activity was measured objectively using accelerometry [14]
daily (for seven days) self-reported snack food consumption was assessed using a participant diary developed and tested in a previous pilot study [8].
Only aerobic fitness at 24 weeks met the conditions for mediation, and was a significant mediator of all treatment outcomes

\subsubsection{Exergames vs standard exercise vs no exercise}
One study \cite{bock2019exercise} compared exercise videogames with standard exercise and control groups. The results showed that participants in the exercise videogame group had higher levels of moderate to vigorous physical activity and greater improvements in health risk indices, such as cholesterol and HbA1c, compared to the other groups. These findings suggest that exercise videogames can be an effective and enjoyable method to promote sustainable physical activity and achieve significant health benefits.


Pasch et all \cite{pasch2009movement} created a theory about the relationship
between body movement and gaming experience using data obtained from interviews, questionnaires, video observations and a motion capture system. 
Two motivations: to achieve and to relax.
Four movement-specific items  to influence immersion in movement-based interaction: natural control, mimicry of movements, proprioceptive feedback, and physical challenge


\subsubsection{Vorzeigestudie}
\cite{ijaz2020player}
45 paritcipants in total tested a VR exercise platform, 23 with a static UI and 22 in a Open World simulation
Precaptured data: Experience with wearables, videogames, exergames, VR and motion sickness
Questionnaires: PENS, IMI + specific questions (eg "Did you experience motion sickness during virtual reality session?"). Seven categories: perceived competence, autonomy, immersion, intuitive controls, future play intention, positive affect, and negative affect.
Immediately after the study "“Would you like to try VR-rides again now? With the follow-up item: If no, why?"
think-aloud technique
heartrate and RR-intervals (interval between r-waves, the largest waves in ecg)
Resutls: Ui group was in average 4:40 +- 3:21 in sim, Open world group 9:19 +- 6:23
Open world better in: reported enjoyment, perceived autonomy, immersion
no significant differences in ratings: competence, intuitive controls, total positive/negative affect
"“I am likely to try VR-Rides again in future” => no differences

UI group: more  physical activity, less autonomy and enjoyment
OpenWorld group: more video game frequenxy, less enjoyment and immersion

Autonomy and immersion correlated strongly with enjoyment, autonomy 0.74 / 0.91 and immersion 0.61 / 0.91 (UI / Open world).

No difference in heart rates
RR Intervals: 1.23 +- 0.20 UI vs 1.37 +-0.24 OpemWorld

Think-aloud:
General: VR environment not smooth, clear goal missing eg "It seems this doesn’t have any objective, right?"
UI: 70\% wondering why not moving
OpenWorld: Praised eg "I really want to go that way"

Exercise-focused vs game enjoyer



% Wie sehr beeinträchtigt Sport die Möglichkeit psychophysiologischer Messungen?
\chapter{Miscellaneous}


\section{Balancing difficulty}
Stach et al showed that heart rate can be effectively used in exergames to adapt the grade of difficulty among player with differing performance levels without affecting gameplay engagement too much \cite{stach2009heart}.
Liu et al \cite{liu2009dynamic} used physiological measured to estimate anxiety in order to adapt the difficulty of the played game in real time. Measures were: cardiovascular activity, including interbeat interval, relative pulse volume, pulse transit time, heart sound, and preejection period; electrodermal activity (tonic and phasic response from skin conductance) and EMG activity (from Corrugator Supercilii, Zygomaticus, and upper Trapezius muscles). Measured anxiety could be matched to subjective anxiety rating with 78 \% accuracy. Also, the players received the game as more challenging and satysfiing as when the difficulty was changed based on the players performances. Additionally, their performance improved and the perceived anciety decreased.

\section{Motion detection}
Full-body motion need to be recognized correctly in order to ensure the desired execution of physical activities \cite{casermanfull}.
When using head mounted displays, a low latency is required to enhance the visual feedback and reduce cybersickness \cite{casermanfull}.
When exergames shall foster body movement, it is critical to avoid shortways in exergames, eg Wii Fit allows playing in by sitting on a couch and just moving the Wii remote \cite{laamarti2014overview}. However, if played with a competetive motivation, players may use all possible shortcuts in order to play the most effective strategies. Rasch showed this \cite{pasch2009movement} in the boxing game of Wii Sports when these type of players reduced the neccessary movement to a minimum while players with the goal of relaxation used their whole body. To examine this, he used a combination of interviews, questionnaires (self-assessment manikin \cite{bradley1994measuring}), video observations and a motion capture
system.
Williamson et al already claimed in 2013 that full body sensing hardware is affordable for the general public \cite{williamson2013full}.
Gerling et al showed that older adults can apply and enjoy motion-based game controls as much as sendatary input, at least in short term for rather simple input necessity \cite{gerling2013movement}.

\section{Avatars}
When players of Nintendo Wii Fit used their ideal self as avatars instead of their current selfs, Jin \cite{jin2009avatars} showed that they experienced greater perceived interactivity.


\section{Impact of training frequency of exergames}
One exercise per week led to no effects and three per week to medium effects while 2 exercises per week resulted in most effects  in most motor and selected cognitive outcomes. These outcomes wer measured by: 6-minute walk test (6MWT), body mass, self-reported physical activity, sleep quality, Berg Balance Scale, Short Physical Performance Battery, fast gait speed, dynamic balance, heart rate recovery after step test and 6 cognitive tests. \cite{hortobagyi2012effects}

\section{Motivation}
Controlling rewards (extrinsic rewards) and controlling communication (explicit instructions,) can undermine intrinsic motivation which is rather supported by informational communication (eg emotional and informational feedback, praise) \cite{ryan1982control}.
"A salient informational event increases intrinsic motivation if it signifies competence and decreases intrinsic motivation if it signifies incompetence."
Fisher (1978) \cite{fisher1978effects} found, in order for intrinsic motivation to be high, it must be supported by high performance and control over performance.



\chapter{Explanations}


\subsection{Cronbach's alpha)}
Internal consistency among one factor, e.g. 3 items (average intercorrelation among the three items)
-1 to 1

\subsection{composite reliability (CR))}
Internal consistency among all factors, e.g. 10 factors and 30 items.
Expected to be higher than Cronbach's alpha
-1 to 1

\subsection{Pearson correlation coefficient}
linear correlation between two sets of data / normalized measurement of the covariance. 
-1 to 1

\subsection{Discriminant validity}
shows you that two tests that are not supposed to be related are, in fact, unrelated









\chapter{Conclusion}

\chapter{Prospect}

