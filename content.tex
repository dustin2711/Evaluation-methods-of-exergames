%*****************************************
\chapter{Introduction}
\label{ch:introduction}
%*****************************************

% \cite{}

\section{Motivation}
Fitness video games, now referred to as exergames, gained a lot of popularity in the last years, mostly due to the development of cheap and small motion sensors. This is not surprising as they enable to combine the healthiness of sports with the motivating aspects of video games.
"There was moderate quality evidence that AVGs can result in benefits for self-esteem, increased energy
expenditure, physical activity and reduced body mass index in children and adolescents who used
AVGs in the home environment" \cite{santos2020active}
increased self-esteem in boys (p <0.05; ES: 0.58; 95\% CI: 0.08-1.07) and reduction of girls’ mental confusion (p <0.05; ES 0.58; 95\% CI 0.15-1.06).\cite{andrade2020effect}
They can adress health problem of video gamers as they are at a higher risk of being overweight \cite{melchior2014internet} or frequent online game players who report exercising at a lower frequency \cite{kowert2014unpopular}. Also elder people with less connections to video games can train their balance using exergames \cite{lai2013effects}. Playing Pokémon Go was shown to be associated with statistically significant number of daily steps \cite{khamzina2019impact}. This becomes more impressive considering that it is the third most popular smartphone game of all time \cite{link_pokemongo}. However, there are also studies coming to the conclusion that offering exergames, instead of regular video games, to children does not increase their physical activity under naturalistic circumstances \cite{baranowski2012impact}. 
To further improve exergames, reliable ways to evaluate them are advantageous. The available research mostly focused on pure videos games and serious games. Serious games are all video games which pursue another "characterization goal" aside from entertainment, e.g. fighting cancer cells in Re-Mission \cite{link_remission} was shown to help children with cancer to develop a more positive attitude towards their illness with positive health aspects, as shown in randomized controlled trials \cite{kato2008video}.
Exergames represent a subgenre of serious games with a focus on excersing i.e. body movements. As serious games benefit, due to their double mission, from specialiced evaluation methods \cite{caserman2020quality} in comparison to general video games, exergames again can benefit from evaluation methods that consider the involved physical activity. This paper will give insights to the challenges and possibilities.

\section{Approach}
Three studies were proposed as interesting by the supervisor Polona Caserman who is an researcher in the field of serious games They acted as a good starting point for snowball sampling.
Other studies were often found using ChatGPT 3.5 with queries in the format of "list some studies about ..." which usually gave a good overview. Although the name of the listed studies were wrong many times, the author name and the release year could be used to find the related paper in Google or Google Scholar. 
E.g. the query "list some studies about to evaluation of exergames" led to these studies \cite{fogel2010effects} \cite{lai2013effects}.
To find evidence for several trains of thought or to answer upcoming questions, key words where put into Google Scholar, e.g. "Heart rate variability emotion" to find more about the connection of heart rate variability and emotions.

\chapter{Measuring outcomes}
"The [metastudy] results show that playing AVGs significantly increased heart rate, VO2 (oxygen consumption), and EE (energy expenditure) from resting" \cite{peng2011playing}

\section{Trainingsdauer}
Messungen:
"6-minute walk test (6MWT), and secondary outcomes were
body mass, self-reported physical activity, sleep quality, Berg
Balance Scale, Short Physical Performance Battery, fast gait
speed, dynamic balance, heart rate recovery after step test,
and 6 cognitive tests."
Anzahl Trainings pro Woche: "no (1×), most (2×), and medium (3×/week) effects of exergaming volume in most motor and selected cognitive outcomes."
\cite{hortobagyi2012effects}

\chapter{Impact of sport on physiological measures}
"HRV measurements change with exercise intensity. For
example, Hottenrott et al. (21) demonstrated that increased exercise intensity was associated with an
increased LF/HF ratio, reflecting increased sympathetic tone in recreational cyclists. Elite triathletes
challenged with moderate-intensity and high-intensity exercise tests showed increased sympathetic
and parasympathetic tone, respectively (14). Similarly, during exhaustive exercise, Pichon et al. (34)
reported increases in parasympathetic input. These studies indicate that, depending on the length
and intensity of exercise, the autonomic input may be sympathetic up to a point with parasympathetic
predominance at exhaustion."
SUMMARY: ShortTerm exercise => increased LF/HF ratio => Sympathethic, LongTerm bei exhaustion => Parasympathetic;
"Section Summary: HRV Changes are Detectable in Overtraining and the Over-Reaching (PreOverTrained) State. Multiple studies illustrate that HRV variables change in the context of
overtraining (summarized in Table 5). Several studies reported changes in HRV variables that reflect
enhanced sympathetic activity at rest in the overtrained state while others identified a predominance
of parasympathetic activity at rest in the overtrained states. The different forms of overtraining that
have been described may explain the changes. Specifically, overtraining associated with increased
exercise volume (sets x repetitions) and intensity (percent repetition maximum) has been associated
with a sympathetic or parasympathetic predominance, respectively (21). Recent studies also provide
evidence that in the early stages of overtraining, designated as overreaching phases, HRV may be
characterized by a sympathetic dominance. When this over-reaching phase occurs over a longer
period of time, then, the development to that of the overtrained stated, as characterized by
parasympathetic predominance, may ensue" 
SUMMARY: Overtraining => Sympathethic, then when exhausted Parasympathetic dominance;
As summarized by Makivic, during exercise, HRV typically shifts toward a higher LF/HF ratio, reflecting increased sympathetic activation and after exhausting excerise, towards a more parasympathetic activation.	\cite{makivic2013heart}

HRV may become less informative during exercise or periods of increased heart rate as root mean square of successive differences and approximate entropy of R-R intervals (ApEnRR), did not significantly differ during exercise\cite{boettger2010heart}

Physical activity leads to increased heart rate \cite{hammond	1985normal, boettger2010heart, haskell2007recommendation}, vigorous activity leads to rapid breathing and substantial increase in heart rate. \cite{haskell2007recommendation}

\section{Smartwatches}
Energy expenditure: Fitbit Surge HR and TomTom watch(r = 0.62-0.69)
average HR: Apple Watch, Fitbit Surge HR, and TomTom watchr (r = 0.47-0.74)
Peak HR:  all smartwatches  (r = 0.59-0.65).  [58 Pope 2019 Validation of four smartwatches in EE and HR]

\chapter{Existing evaluation methods}
\label{ch:evaluationmethods}

% \section{Overview}

There are multiple methods available to evaluate games with different characteristics. The methods can be applied during or after the gaming sessions. They can be subjectively, measured by a human and often the players themselves, or objectively using a device. These distictions create the following table.
 
\begin{table}[]
	\begin{tabular}{llll}
					& Live                                              & Afterwards\\
	Subjective      &                                                   & \\
	- Self-reported & Live questionnaires, think-aloud                  & Questionnaires, interviews,\\ 
					& 													& focus groups, heuristic evaluation \\
	- Observed      & Behavioral observation 							& \\
	Objective       & Physiological measures, 							& \\  
					& eye tracker, game metrics         				&\\                                                    
	\end{tabular}
\end{table}                                             								


\section{Questionnaires}
One important part to understand player's impressions towards video games are questionnaires as they can gather large amounts of data in a standardized way. One major drawback is that they need some time to complete and thus, are often conducted some time after the game was played. The Game Experience Questionnaire (GEQ) tries to solve this problem using a more concise In-game questionnaire \cite{poels2007game}. 

Another validation study use exploratory and confirmatory factor analyses to find partial support for the GEQ and large support for the PENS \cite{johnson2018validation}, also proposing some revisions to the factor structures
GEQ: negative affect, tension/annoyance and challenge => negativity;
	intuitive controls, competence => competence
PENS: 3 categories presence, autonomy and relatedness supported



\subsection{The Player Experience of Need Satisfaction (PENS) Model}
The creators of this questionnaire claim that fun and satisfaction are outcomes of psychological processes and not the processes themselves \cite{rigby2007player}. So in order to create an entertaining game for the player, you need to unterstand the underlying processes und be able to describe the "underlying energy that fuels actions" \cite{rigby2007player}. Thereby, PENS is grounded on the well-established self-determination theory from the 1980s. Ut elicited validated measures in many fields \cite{pietrabissa2020development,lohmann2017measuring,richards2021further}. This theory suggests that there are three basic psychological needs \cite{deci1985intrinsic}:
\begin{itemize}
	\item Autonomy: The need to feel in control of one's actions and have the freedom to make choices.
	\item Competence: The need to feel capable, effective, and skilled in one's pursuits or endeavors.
	\item Relatedness: The need to feel connected, supported, and engaged in meaningful relationships with others.
	\item -
	\item Presence/Immersion: emotional engagement in the game
	\item Intuitive Controls
\end{itemize}


\subsection{Game Experience Questionnaire (GEQ)}
The Game Experience Questionnaire was developed 2007 by a European research project named FUGA \cite{law2018systematic}. It is widely used in at least 515 papers \cite{law2018systematic}. Due to its importance, its story shall be presented. 
To create the questionnaire, FUGA first formed focus groups of both frequent and infrequent gamers with three questions to come up with reasons for, feelings when and feelings after playing video games. Then, five experts including psychologists and experts on measurement development met. They used theoretical considerations as well as the focus group's results to create a model to understand gaming experience. They came up with ten factors that were used to create a questionnaire with 92 items and a 5-point Likert-scale from "not at all" (0) to "extremely" (4).
Then, 380 participants were recruited online to test the questionnaire with a game of their choice. Then a exploratory factor analysis was run on the results. They came up with a 7-factor solution with 82 items. It explained 52 \% of variance in all items. Most questions have more than 30 \% correlation to only one factor. The seven factors are:
\begin{enumerate}
	\item Immersion (previously two factors Sensory and Imaginary Immersion)
	\item Tension (said to unforseenly emerged from Negative Affect although the factor Suspense already existed)
	\item Competence (includes previous factor Experienced Control)
	\item Flow
	\item Negative Affect
	\item Positive affect
	\item Challenge
\end{enumerate}
The factors "Connectedness" and "Negative affective experiences related to playing with others" were shifted to an own questionnaire, the "social presence module" including another new category "behavioural involvement" 

\subsubsection{Systematic Review and validation of the GEQ}
A paper from 2018 did a systematic review using 73 studies that applied the GEQ \cite{law2018systematic}. Many critic points were found for both a GEQ itself but also for the studies that used it 
\begin{itemize}
	\item Only 31 papers of 73 offered an explanation why the GEQ was used (31 because it is validated, 10 due to popularity, 8 due to multidimensional structure and 6 for being theoretically and empirically founded) 
	\item 47 Papers did not state the number of questions used whiel the GEQ offers a 33- and a 42-items version 
	\item Often times, papers used a selection of factors or items or adapted the items with mostly missing reasoning, partially explained by overlapping after using factor analysis 
	\item The used scale was not reported by 40 papers, 27 did not use the origin scale and 3 of them even used a mix of 5 and 6 answer options instead of 5 
	\item One main critic point was that there are two versions of the GEQ, from 2007 and 2013 with none of them formally publicated. The 73 analyzed papers referenced the 2007 version only once and the 2013 version not a single time. Often times, the GEQ was cited as "Manuscript in preparation" 
	\item 17 papers stated Cronbach's $\alpha$ for internal consistency of the factors with following values:
	\begin{itemize}
		\item Flow and Competence: 0 7 to 0 94
		\item Positive Affect and Immersion: 0 49 to 0 85
		\item Negative Affect, Challenge and Tension: 0 3 to 0 74
	\end{itemize}
12 of these 17 papers stated low internal consistency, especially for challenge and negative affect 
No paper could reproduce the 7-factor structure but six proposed using six factors with Negative and Positive Affect being summarized as "joy" 
\end{itemize}

Finally, a validation study was conducted with 633 participants in an online survey (age = 33 47 ± 10 57, game experience = 19 5 ± 8 9 years) using the 33-items GEQ. A confirmatory factor analysis resulted in the factor Negative Affect not being satisfactory due to a low reliability coefficient and the factor Challenge just barely. Also, a exploratory factor analysis was conducted. It came to the conclusion to remove some items that did not load well on the given or any other factor. Flow may be renamed to "loss of time". The Challenge factor only has 3 remaining items with need for filling up. Also Tension and Negative Emotion should be combined to one factor which results in these factors:
\begin{enumerate}
	\item Immersion
	\item Competence
	\item Flow
	\item Negative Affect
	\item Positive affect
	\item Challenge
\end{enumerate}

Unfortunately, the reviewing paper does not adress multiple issues.
What are possible readons why the results of the conducted study differ from the original study?
Are there maybe missing questions that could lead to new factor that the GEQ is missing?
Can the items of Negative and Positive Affect be summarized like in six reviewed papers? (with inverting the questions for Negative Affect)
Can the factor Challenge be added to the factor Competence with inverted questions?

\subsection{Flow State Scale}
\hint{More sources to be included }
Flow describes a positivly experienced state, where a person is totally in a task with challenge and skill level holding a perfect balance \cite{jackson1998psychological}. Further research has shown that, when experiencing flow, the perceived challenge level is even slightly higher than the skill level (8 3 vs 7 3) \cite{jackson1996development}. Getting in flow state would be pleasant for exergames too as people seem to show peak performance while in flow state \cite{jackson1996development}. It also plays a role in motivation and enjoyment \cite{jackson1996development} 
To evaluate flow, there is a questionnaire called the "Flow State Scale". It originally consisted of 54 items with 9 factors but could be reduced to 36 items while keeping a high reliabiliy with coefficient alphas rated above 0 80 \cite{jackson1996development} 
The 9 original factors are \cite{jackson1996development}:
\begin{enumerate}
	\item Sense of Control
	\item Challenge-Skill-Balance
	\item Clear Goals
	\item Merging with task
	\item Unambigious feedback
	\item Concentration
	\item Loss of Self-Consciousness
	\item Transformation of Time (potential for removal as weakest factor)
	\item Intrinsic motivation (potential for removal as it may be more of a higher-order factor)
\end{enumerate}

\subsection{Player Experience of Need Satisfaction (PENS)}
\hint{To be done}


\section{Psychophysiological measures}

\subsection{applicability in Videos games}
" Some of the earliest are from Mandryk and
Inkpen (2004) and Hazlett (2006), who have presented studies (albeit with
small sample sizes) supporting the use of psychophysiological measures in
game research. More recently, Nacke (2009) and others have published studies as an attempt at a common methodology for a design-oriented approach.
Their papers evaluate EEG (Nacke et al. 2010b), EDA, HR (Nacke 2009;
Drachen et al. 2009) and facial EMG (Nacke 20" \cite{kivikangas2011review}
"clearly showed the necessity of proper
experimental design and that care must be taken in interpreting the signals:
otherwise, for instance, self-reported and physiologically indexed emotions
may turn out to assess significantly different things"  \cite{kivikangas2011review}
"All these
studies demonstrate that physiological signals are closely related to players’
self-reported emotional states and behaviour, while Chanel and others (2011)
found support that the fusion of several physiological modalities increases the
recognition accuracy. This shows that multiple measurements are still needed
for a reliable interpretation of the player experience"  \cite{kivikangas2011review}





ToDo: Paper 54 lesen




Psychophysiological measures can be a helpful way in evaluating exergames because they " provide an objective, continuous, real-time, noninvasive, precise and sensitive way to assess the game experience" \cite{kivikangas2011review}. 
"A large number of studies have shown that psychophysiological measures can be used to index emotional, motivational and cognitive responses to media messages" \cite{kivikangas2011review}
"physiological signals could be used, for example, at the player test phase to identify strong emotional
episodes and compare them to expectations, or to control the successful emotional elicitation of game event designed to be emotionally arousing" \cite{kivikangas2011review}
"Psychophysiological research is defined as using physiological signals to study
psychological phenomena (Cacioppo et al. 2007: 5)." \cite{kivikangas2011review}
The human body offers many ways to gain insights into someone's emotions state, e.g. heart rate and skin conductance together can be used to effectively recognize some basic emotions \cite{hamdi2015emotion}. Also, these measurements can be taken continously during the gameplay and can be mapped to a recorded gameplay session afterwards \cite{nacke2015physiological} which allows great post-analysis. Nonetheless, they should not be relied on alone. The researcher should always ensure the correct interpretation of the signals \cite{nacke2015physiological}, e.g. by validating them using other measurement methods like questionnaires 
One major drawback of these measures is that they are not only be affected by the game experience itself. Any movement with the body affects physiological measures, e.g. physical exercise results in heart rate increase \cite{javorka2002heart}. This can be obviously a problem in evaluating exergames where the body response usually consists of a psychological and a physical reaction. Aside from moving, also any stimulus in the environment that is not gameplay-related can be disturbing and affect the meausrement \cite{nacke2015physiological}. Also the demographic background of any person and the baseline of the measured variable in idle state should be taken into consideration \cite{nacke2015physiological} 
Although it is often hard to directly map physiological measurements to basic emotions, Russell’s circumplex model can be applied to simplify things. It uses just two dimensions valence ("pleasant/unpleasant emotion") and arousal ("low/high stimulation"). Positive and negative emotional valence can be assessed by using facial EMG \cite{kivikangas2011review}.They can be more easily measured and concrete emotions can be derived from them \cite{seo2019automatic} 
This is why many measuring methods are often not applicable for exergames which will be handled in the next chapter 
Another important point for exergames is portability and applicability during excercise. There may be games with heavy movement where a small or wireless measurement device may be advantageous 
Now comes an overview of measure possibilities. For more details, the just mentioned paper by Nacke 2014 is recommended 

\subsection{Facial Electromyography (EMG)}
Sensors will be attached to the skin and measure the eletric and such physical activity of muscles. They can be used to detect facial expressions. It was shown that eye \cite{ravaja2018phasic} and eyebrow muscles are a good indicator for positive and cheek muscles a good indicator for negative emotions \cite{nacke2015physiological,mandryk2006using}. 
Disadvantages are discomfort wearing face sensors, the inability to talk during gameplay and unnatural facial expressions due to feeling the sensor \cite{nacke2015physiological}
Also computer vision techniques can be used to obtain facial expressions but when discovering smiling, "The results showed that EMG has the advantage of
being able to identify covert behavior not available through
vision. Moreover, CV appears to be able to identify visible
dynamic features that human judges cannot account for" \cite{hernandez19invisible}

\subsection{Electrodermal Activity (EDA) / Skin conductance}
Two electrodes measure the conductivity of the skin. This is correlated to the sweat gland activity and that again is linked to emotional arousal \cite{nacke2015physiological, dawson2000electrodermal}. The measuring device can function wireless and be made out of soft materials \cite{kim2021soft} do not be disturbing during exercise 
The main disadvante is some seconds of latency which makes it more difficult to link a signal to a cause \cite{nacke2015physiological} 

"Significant covariation was obtained between (a) facial expression and affective valence judgments and (b) skin conductance magnitude and arousal ratings" \cite{lang93pictures}

\subsection{Cardiac activity / Heart Rate}
Heart rate was shown to effectively recognise all four basic negative emotions anger, fear, disgust and sadness \cite{levenson2003blood} and is tied to emotional arousal \cite{nacke2015physiological}. There are many possibilities to measure heart rates. Chest strap monitors are very precise and are affordable (75 €) \cite{link_herzfrequenzsensor} and also current smartwatches can deliver good results \cite{gilinov2017variable}.

\subsection{Heart Rate Variability}
The variability of the heart rate can be also obtained from measuring the heart rate but needs further analysis \cite{nacke2015physiological}. There is emerging analysis for its role in regulated emotional responding \cite{appelhans2006heat} with higher variability typically correlating with better emotional regulation [source?]. However, it was shown that heart reate variability alone is not that effective in recognizing emotions with a baseline below 50 \% \cite{ferdinando2014emotion}.

\subsection{Electroencephalography (EEG)}
Electroencephalography is used to measure brain waves with a high temporal accuracy \cite{nacke2015physiological}. The activity of different brain regions can be visualized but there is a lot of interpreation possibilities, especially since the origin of the brainwaves is not apparent \cite{nacke2015physiological}. At least it can be used to measure a player's engagement in a video game, e.g. there was increased activity when diing in. Super Meat Boy \cite{mcmahan2018evaluating}. With the Emotive EEG headset \cite{link_emotiv}, there is also portable technology available.
"Relatively greater left frontal activity indicates a propensity to approach a stimulus, whereas relatively greater right frontal activity indicates a propensity to withdraw from a stimulus (Davidson 1998). It must be emphasized that
frontal asymmetry is not a measure of positive or negative affect per se, but
it taps a broader motivational tendency towards approach-related or withdrawal-related behaviours and emotions (Allen et al. 2001; Davidson 1998)" \cite{kivikangas2011review}

\subsection{Respiration rate}
Studies reported higher respiration rate during arousal \cite{homma2008breathing}, increases in inspiratory time when laughing and decreases during disgust \cite{boiten1998the}. Although this measure does not give as much insights as others, it can be well estimated from using heart rate sensors, so no addtional hardware is neccessary when heart rate is already measured. The root mean square error then is 0.648 per minute \cite{natarajan2021measurement}.

\subsection{Eyes}
"The best accuracy achieved by fuzzy integral fusion strategy is 87.59\%, whereas the accuracies of solely using eye movements and EEG data are 77.80\% and 78.51\%, respectively." to detect "positive, negative and neutral" emotions. Uses 33 different features just from the eyes (pupil diameter, dispersion, fixation, blinking, saccade (ruckartiges Verhalten))\cite{lu15combining}

\subsection{Body}
"To extend from psychophysiological measurements, there is some evidence that body movement and position (measured by acceleration sensors or position cameras) might be associated with
attention, interest and emotions (FUGA Contributors 2009; Kivikangas and
Ravaja in preparation)." \cite{kivikangas2011review}
]
\section{Comparing exergames with conventional interventions}

The effectiveness of exergames can be evaluated by comparing their outcomes with conventional interventions. Many evaluations methods and dependant variables are conceivable 

\subsubsection{Physical activity in inactive children}
A study from 2010 has shown the big potential of exergames for a physical education classroom \cite{fogel2010effects}. By the teacher's observation, four 5th grade class children did not move as much as other children. These "inactive" labeled children should be motivated to move more and participated in the study. 
Contrary to the default lesson where children played e.g. ball games, the exergame lesson consisted of 11 exergame stations. There, the children rotated every 10 minutes and played games like "Fit Interactive 3 Kick", a martial art simulator that uses three foam pads to be hit 
The measured variable was "physical activity" which was defined as moving a large muscle group. It was shown that the inactive children increased their physical activity from 1 6 min to 9 2 min during a lesson. This massive increase was only made possible by enhancing the time of opportunities for participating in excercises from 3 8 to 11 6 min. This was mainly reasoned by the fact that the teacher spent way less time providing instructions. 
It is unclear if the effects would persist longer than the given six weeks. Addtionally, the study relies on one single physiological measure as it spared measuring the heart rate in favor of saving time during the lessons. Still, the pure method of comparing exergames with a default setting enabled interesting insights and should be further examined.

\subsubsection{Balance training for elder people}
A study from 2012 evaluated the effectiveness of exergame exercises on balance improvement in older adults \cite{lai2013effects}. After six weeks, significant improvements were observed in typical balance tests such as the Berg Balance Scale. The positive effects partially persisted even after 6 weeks without further exercises. The exergame approach showed advantages in terms of feasibility and attractiveness compared to conventional exercises, due to instant and positive feedback and the more challenging nature. The study suggests the need for further research to examine the long-term effects and compare the exergame approach with traditional physical therapy.

Riding a bike while watching a bike tour video that gets played faster the faster you cycle was compared with riding a bike to play a video game with way more motivation for the video game. \cite{hardy2011adoption}


\section{Evaluation of excersises}


\chapter{Evaluation methods for exergames}

% \chapter{Test chapter}
% \label{ch:evaluation}

% \section{Title}

% \begin{table}[htb]
% \centering
% 	\begin{tabular}{ll}
% 	    \toprule
% 		\textbf{Parameter} & \textbf{Value} \\
% 		\midrule
% 		P1 1 & V1 1 \\
% 		P1 2 & V1 2 \\
% 		P1 3 & V1 3 \\
% 		\midrule
% 		P2 & V3 \\
% 		\midrule		
% 		P3 & V4 \\
% 		\bottomrule
% 	\end{tabular}
	
% 	\caption{Evaluation Parameters}
% 	\label{tab:eval_params}
% \end{table}

% \begin{figure}[htb]
% 	\centering	
% 	\subcaptionbox{Alternative Serious Games Circle 1 %
% 		\label{fig:example_2}}%
%  		[ 48\linewidth]{
%   		\includegraphics[width=0 35\textwidth]{gfx/sg-smiley png} 	
%   	}  	
%   	~
%   	\subcaptionbox{Alternative Serious Games Circle 2 %
%   		\label{fig:example_3}}
%  		[ 48\linewidth]{
%   		\includegraphics[width=0 35\textwidth]{gfx/sg-circle_improved png}  	
%   	}	  
  		
% 	\caption{Alternatives of Serious Games circle }
% 	\label{fig:ex_2_3}
% \end{figure}